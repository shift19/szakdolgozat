\Chapter{Példák ábrák szerkesztésére}

% Kb. 6 oldal

% 4-5 példával bemutatni, hogy milyen esetekre hogyan használható az elkészült szerkesztő.

% Jó, hogy ha változatosak a példák, tehát lehet benne Hasse diagram, általános gráf, egyenletek, függvények, folyamatábra.
