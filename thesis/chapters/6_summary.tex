\Chapter{Összefoglalás}

A szakdolgozatom egy olyan online grafikus szerkesztő létrehozása volt, amellyel \LaTeX\-be visszailleszthető kód hozható létre. 

A dolgozat elkészítése közben alaposabban megismerhettem a Bootstrap, a KaTeX, és a p5.js függvénykönyvtárak működését, előnyeit, hátrányait, és hiányosságait. Ha újra abban a helyzetben lennék, hogy válogatni kellene a könyvtárak közül, biztos vagyok benne, hogy ugyanúgy ezeket választanám. A különböző könyvtárak dokumentációja az online felületein is elérhetők, interaktív példákkal illusztrálva a működésüket.

Véleményem szerint sikerült egy olyan webes alkalmazást létrehoznom, amely beleillik a mai modern weboldalakba. A célom az volt, hogy egy egyszerű, könnyen használható, azonban mégis funkciókban gazdag alkalmazás készüljön el.



A szoftverfejlesztésben még mindig rengeteg lehetőség és esély rejlik. Az alkalmazásom esetében az egyik, amit kiemelnék ezek közül a \LaTeX\ forráskód betöltése, ugyanis jelenleg minden elkészült ábrához tartozik egy kódolt karaktersorozat, amely tartalmazza az információkat, melyből az adott alakzat visszatölthető.

Abban egészen biztos vagyok, hogy a szakdolgozat elkészítése során szerzett rengeteg hasznos ismeretet alkalmazni tudom majd a jövőben is.