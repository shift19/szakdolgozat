\Chapter{Bevezetés}

A szakdolgozatom egy online grafikus szerkesztő elkészítése és dokumentálása. A szerkesztő a \textit{TikZ} \LaTeX\ csomag nyelvi elemeire épül. A webes fejlesztés miatt a HTML5 és a JavaScript nyelv adja az alapokat. A rajzolási felületet a \textit{p5.js} nyújtja.

A \LaTeX\ -et széles körben használják a tudományos életben tudományos dokumentumok közlésére és közzétételére számos területen, többek között a matematikában, az informatikában, a mérnöki tudományokban, a fizikában, a kémiában, a közgazdaságtanban, a nyelvészetben, a kvantitatív pszichológiában, a filozófiában és a politikatudományban. A \LaTeX\ a \TeX\ szövegszerkesztő programot használja a kimenet formázásához, és maga is a \TeX\ makrónyelvben íródott.

A tanulmányaim során megismerkedtem a JavaScript nyelvvel és egy könnyen elsajátítható programozási nyelvnek találtam. A JavaScript, gyakran JS rövidítéssel, az ECMAScript specifikációnak megfelelő programozási nyelv. A JavaScript magas szintű, gyakran futásidejű fordítású nyelv. Az ECMAScript 2015 (\textit{ES6}) bevezetésével egy jól használható  objektumorientált nyelv lett.

A következő oldalakban megismerkedhetünk a TikZ csomag telepítésével, a grafikus elemivel, valamint a már meglévő szerkesztők kerülnek jellemzésre, majd összehasonlításra különböző szempontok alapján. Ezek a feltételek kerülnek megfogalmazásra a készülő alkalmazással szemben, mint követelmények.

 A következő szegmens már a dokumentáció része az alkalmazásnak. Itt kerülnek kifejtésre a felhasznált függvénykönyvtárak, az elkészült alkalmazás felépítése, funkciói, használata, és a definiált osztályok leírása, a \textit{TikZ}, mint \LaTeX\ könyvtár és a \textit{p5.js}, mint JavaScript függvénykönyvtár eszközkészletében található eltérések. 
 
 A továbbiakban már az elkészült alkalmazással megvalósított ábrák kerülnek bemutatásra.

