\pagestyle{empty}

\noindent \textbf{\Large A mellékelt CD tartalma}

\vskip 1cm

\noindent A dolgozathoz tartozó melléklet a következőket tartalmazza:

\begin{itemize}
\item \texttt{thesis.pdf}: a dolgozat PDF formátumban
\item \texttt{manual.pdf}: a használati útmutató PDF formátumban
\item \texttt{thesis/}: a dolgozat \LaTeX\ forráskódját tartalmazó jegyzék
\item \texttt{editor/}: az elkészített szerkesztő jegyzéke
\end{itemize}

\vskip 0.5cm

\noindent
A szerkesztő futtatásához egy webszerver szükséges. Amennyiben nem áll rendelkezésre akkor egy \textit{Node.js} telepítése az egyik legegyszerűbb megoldás. Ez letölthető az alábbi linken:

\vskip 0.5cm
\noindent
\href{https://nodejs.org/en/download/}{\texttt{https://nodejs.org/en/download/}}

\vskip 0.5cm
\noindent
A \textit{http-server} telepítése:

\begin{lstlisting}[language=bash]
>   npm install --global http-server
\end{lstlisting}

\vskip 0.5cm

\noindent
Az alkalmazás elindításához ezenkívül egy parancssorra lesz szükség:

\begin{lstlisting}[language=bash]
>   http-server ./editor/
\end{lstlisting}

\vskip 0.5cm

\noindent
A parancs kiadása után már el is érhető al alkalmazás, jelen esetben a \textit{8080}-as port alatt:

\begin{lstlisting}[language=bash, alsoletter=0123456789, morekeywords={visible, none, disabled, seconds, 3600, 120, 8080}]
>   http-server ./editor/
Starting up http-server, serving ./editor/

http-server version: 14.0.0

http-server settings:
CORS: disabled
Cache: 3600 seconds
Connection Timeout: 120 seconds
Directory Listings: visible
AutoIndex: visible
Serve GZIP Files: false
Serve Brotli Files: false
Default File Extension: none

Available on:
http://192.168.0.2:8080
http://127.0.0.1:8080
	
Hit CTRL-C to stop the server
\end{lstlisting}
