\Chapter{A TikZ és eszközkészlete}

% Kb. 8 oldal

\Section{Ábrák szerkesztése LaTeX-ben}

\Section{A TikZ elemei}

\SubSection{Használata}

% usepackage, telepítés, útmutatók

\SubSection{Szintaxis}

% Alapvető nyelvi elemek bemutatása

\SubSection{Elérhető diagram elemek}

% Csomópontok, vonalak, feliratok, ívek, ...
% Ide kerülhetnek külön ábrák részletes magyarázatokkal (ténylegesen TikZ-s ábrák itt is).

\Section{Szerkesztőeszközök}

\SubSection{Editor 1 ...}

% Sorban be kell mutatni, hogy milyen szerkesztőeszközök vannak.

% Képernyőképek

% Összehasonlító táblázat.
% Az összehasonlítás történhet a 3. fejezetben felsorolt szempontok szerint például.
